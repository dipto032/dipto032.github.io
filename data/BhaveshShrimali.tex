\documentclass[10pt]{article}
\usepackage[margin=0.4in]{geometry}
\usepackage[utf8]{inputenx}
%\usepackage[T1]{fontenc}
\usepackage{lmodern}
\usepackage{amsmath,amsfonts,amssymb}
\usepackage{float,subcaption,enumitem}
\usepackage{graphicx}
\usepackage{hyperref}
\usepackage{cleveref}
\usepackage[dvipsnames]{xcolor}
\usepackage{array,booktabs}
\usepackage{fontawesome}
\usepackage{fancyvrb}
\usepackage{bibentry}
\hypersetup{
	colorlinks,
	linkcolor={blue!50!black},
	citecolor={blue!50!black},
	urlcolor={blue!80!black}
}
\newcommand{\myname}[1]{{\color{blue}\bf{#1}}}
\pagenumbering{gobble}
%\renewcommand{\familydefault}{\sfdefault}
\setlength{\parindent}{0pt}
\setlength{\parskip}{0pt}
%\usepackage[letterspace=10]{microtype}
\begin{document}
%	\lsstyle
\nobibliography{references}
\bibliographystyle{unsrt}
{\Large \bf %\sffamily 
	Bhavesh Shrimali}\\[-2mm] \rule{1.02\textwidth}{.5pt}\vspace{3mm}
\begin{minipage}[t]{0.15\linewidth}
	{\textsc{Contact}} \\ \textsc{Information}
\end{minipage}\hfill
\begin{minipage}[t]{0.50\linewidth}
	B150, Newmark Civil Engineering Laboratory\\
	University of Illinois Urbana-Champaign
\end{minipage}\hfill
\begin{minipage}[t]{0.3\linewidth}
\begin{flushright}
	%\scriptsize
	\href{mailto:bhavesh.shrimali@gmail.com}{\faEnvelope\ : \textsf{bhavesh.shrimali@gmail.com}}\\
	\href{https://bhaveshshrimali.github.io}{\faGlobe\ : \textsf{bhaveshshrimali.github.io}}
\end{flushright}
\end{minipage}\\[3mm]
\begin{minipage}{0.15\linewidth}
	\textsc{Research}\\ \textsc{Interests}
\end{minipage}\hfill 
\begin{minipage}{0.825\linewidth}
homogenization; applied math; numerical analysis; high performance computing;\\ variational models of fracture; finite element method
\end{minipage}\hfill\\[3mm]
\begin{minipage}{0.15\linewidth}
	\textsc{Education}
\end{minipage}\hfill 
\begin{minipage}[t]{0.825\linewidth}
	{\bf {University of Illinois Urbana-Champaign}}\hfill {\bf 4.0/4.0}
	\begin{itemize}[label={},topsep=0.5mm,itemsep=-2pt,itemindent=-1.2em]
	\item MS and Ph.D., Structural Engineering and Mechanics, 2015-present\\ \textsl{Advisor: Prof. Oscar Lopez-Pamies}
	\item Minor: Computational Science and Engineering, 2016-present
	\end{itemize}\vspace{1ex}
	{\bf Indian Institute of Technology Guwahati}\hfill {\bf 9.22/10.00}
	\begin{itemize}[label={},topsep=0.5mm,itemsep=-2pt,itemindent=-1.2em]
		\item 	B.Tech, Civil Engineering, 2011-15
		\item \textsl{Ranked $1^{\rm st}$ in the department}
	\end{itemize}
\end{minipage}\\[4mm] 
%\begin{minipage}[t]{0.2\linewidth}
%\begin{flushright}
%	{\bf 4.0/4.0}
%	\\~ \\~ \\[2mm]~\hfill	{\bf 9.22/10.00} \\
%\end{flushright}
%\end{minipage}\\[4mm]
\begin{minipage}{0.15\linewidth}
	\textsc{Publications}
\end{minipage}\hfill
\begin{minipage}[t]{0.75\linewidth}
	\vspace{-1em}
	\begin{enumerate}[label={[J\arabic*]},leftmargin=-3em]
		\item \bibentry{BhaveshJMPSPorous}\hfill[\href{http://pamies.cee.illinois.edu/Publications_files/JMPS2019.pdf}{\bf\textsf{pdf}}]
	\end{enumerate}
\end{minipage}\\[3mm]
\begin{minipage}[t]{0.15\linewidth}
	\textsc{Awards}\\ \& \\ \textsc{Achievements}
\end{minipage}\hfill
\begin{minipage}[t]{0.85\linewidth}
	\vspace{-1em}
	\begin{itemize}[itemsep=-1pt]
		\item {\bf Institute silver medal}, IIT Guwahati: Ranked first in Civil Engineering Department\hfill [\textsl{'15}]
		\item {\bf DAAD-WISE fellowship}: Awarded to 150 students from all over India to pursue a summer research internship in Germany \hfill [\textsl{'14}]
		\item {\bf IET India Scholarship}: Awarded to about 170 students from select universities \hfill [\textsl{'14}]
		\item {\bf HKUST IIP}: Selected for the International Internship Program (IIP) to pursue a summer research internship at HKUST \hfill [\textsl{13}]
		\item {\bf OPJEMS}: Selected among the top 50 students nationwide for the prestigious scholarship \hfill [\textsl{'13}]
		\item {\bf Dhrishti 2012}, India: Among the \textsl{top 5} teams in India based on proposed novel engineering solutions to renewable energy generation\hfill [\textsl{'12}]
		\item {\bf Institute Merit Scholarship}, IIT Guwahati: Awarded annually to the top ranked student in each department based on year round academic performance\hfill [\textsl{'12-'15}]
		\item {\bf Indian National Olympiads}: Selected for the national rounds of olympiads in Mathematics, Physics and Astronomy\hfill [\textsl{'11}]
		\item {\bf \href{http://www.kvpy.iisc.ernet.in/main/index.htm}{KVPY Fellow}}: Ranked among \href{http://www.kvpy.iisc.ernet.in/main/fellows.htm#2009}{209} students from all over the country \hfill[\textsl{'10}]
		\item {\href{http://www.ncert.nic.in/programmes/talent_exam/index_talent.html}{\bf NTSE Scholar}}: Ranked 324/1000 in the country \hfill [\textsl{'07}]
	\end{itemize}
\end{minipage}\\[3mm]
\begin{minipage}{0.15\linewidth}
	\textsc{Teaching}
\end{minipage}\hfill
\begin{minipage}[t]{0.85\linewidth}
	\vspace{-1em}
	\begin{itemize}[itemsep=-1pt]
		\item {\bf CEE 598: Constitutive Modeling of Engineering Materials} \hfill [\textsl{Spring '19}]
		\item {\bf CEE 570: Finite Element Methods} \hfill [\textsl{Spring '18}]
		\item {\bf CEE 471: Structural Mechanics*}: \href{https://citl.illinois.edu/docs/default-source/teachers-ranked-as-excellent/tre-2017-fall.pdf}{List of teachers ranked excellent by UIUC.}\hfill [\textsl{Fall '17, '18, '19}]
		\item {\bf CS 357: Numerical Methods} \hfill [\textsl{Spring '17}]
		\item {\bf CS 125: Introduction to Computer Science} \hfill [\textsl{Spring '16, Fall '16}] 
	\end{itemize}
\end{minipage}\\[3mm]
\begin{minipage}{0.15\linewidth}
	\textsc{Technical Skills}
\end{minipage}\hfill
\begin{minipage}[t]{0.85\linewidth}
	\vspace{-1.5em}
	\begin{itemize}[itemsep=-1pt, label={}]
		\item {\bf Languages: }Python, C, C++, Fortran
		\item {\bf FE Codes: }FEniCS, Firedrake, Abaqus
		\item {\bf Miscellaneous: }Bash, Git, PyCUDA, pybind11
	\end{itemize}
\end{minipage}\\[3mm]
\begin{minipage}{0.15\linewidth}
\textsc{Research\\ Internships}
\end{minipage}\hfill
\begin{minipage}[t]{0.85\linewidth}
\vspace{-1.5em}
\begin{itemize}[itemsep=1pt, label={}]
	\item {\bf Technische Universit\"{a}t Braunschweig}\hfill [\textsl{Summer '14}]\\
	\textsl{Advisor: Prof. Dr. Klaus Thiele}\\
	I worked on linear perturbation analysis of a cable stayed bridge to identify the dominant eigen-modes. Subsequently, based on a FE analysis of the galloping cables, I proposed an optimum location of an eddy current damper. The entire set of calculations were done in FE simulation program ANSYS. 
	\item {\bf Hong Kong University of Science and Technology}\hfill [\textsl{Summer '13}]\\
	\textsl{Advisor: Prof. Christopher K Y Leung}\\
	I worked on tuning the proportion of elastomeric and steel fibres in high performance concrete to enhance the post peak softening response. I also worked on validating a FE analysis of a notched beam subject to 3-point bending with the in situ experiments.
\end{itemize}
\end{minipage}\\[4mm]
\begin{minipage}{0.15\linewidth}
	\textsc{Courses}
\end{minipage}\hfill
\begin{minipage}[t]{0.85\linewidth}
	\vspace{-1em}
	\begin{itemize}[itemsep=-1pt, label={}]
	\item Fast Algorithms and Integral Equations, Multigrid Methods, Nonlinear Finite Elements, Generalized Finite Element Method, Numerical methods for PDEs, Computational Inelasticity
	\end{itemize}
\end{minipage}
\end{document}